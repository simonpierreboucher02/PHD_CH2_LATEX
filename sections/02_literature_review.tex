%% Auteur: Simon-Pierre Boucher, Université Laval, Chapitre 2 : thèse de doctorat

\section{Literature Review} \label{sec:literature}

This section provides a comprehensive review of the theoretical and empirical literature relevant to volatility transmission in commodity ETF markets. We organize our discussion around five key themes: the theoretical foundations of ETF arbitrage mechanisms, the empirical evidence on ETF impact on underlying asset markets, the development of high-frequency volatility measurement and modeling techniques, the specific characteristics of commodity markets and their financialization, and the emerging literature on ETF market microstructure during crisis periods. This comprehensive framework allows us to identify critical gaps in existing knowledge and precisely position our contribution within the broader literature on ETF market dynamics and volatility transmission.

\subsection{Theoretical Foundations of ETF Arbitrage Mechanisms}

The theoretical underpinnings of ETF pricing rest fundamentally on the arbitrage relationship between ETF shares and their underlying Net Asset Values, first rigorously analyzed by \citet{ackert2000arbitrage}. Their seminal framework demonstrates that authorized participants (APs) can create or redeem ETF shares in exchange for baskets of underlying securities, theoretically ensuring price alignment through arbitrage. This mechanism relies on the assumption that arbitrageurs can costlessly and instantaneously exploit price discrepancies, maintaining the no-arbitrage condition that ETF prices equal their fundamental values.

However, the theoretical elegance of this framework faces significant challenges in practice. \citet{pontiff1996costly} provides early theoretical insights into the limits of arbitrage, showing that transaction costs, capital constraints, and noise trader risk can prevent arbitrageurs from eliminating pricing anomalies. These insights prove particularly relevant for ETF markets, where authorized participants face substantial operational complexities and regulatory constraints.

\citet{gromb2010limits} extend this theoretical framework by examining how funding liquidity affects arbitrage effectiveness. Their model demonstrates that when arbitrageurs face binding capital constraints, price deviations can persist even when arbitrage opportunities are clearly identified. This theoretical insight has profound implications for ETF pricing, particularly during periods of market stress when authorized participants may face increased margin requirements and reduced access to funding.

The theoretical literature has also examined the role of market makers in ETF pricing. \citet{hendershott2013relationship} develop a model showing that market makers' inventory management concerns can create systematic patterns in ETF pricing errors. When market makers accumulate large inventory positions, they adjust quotes to encourage trades that reduce their exposure, potentially creating persistent deviations from NAV. This theoretical framework helps explain why ETF pricing errors may exhibit predictable patterns related to trading volume and market volatility.

Recent theoretical advances have focused on the feedback effects between ETF trading and underlying asset markets. \citet{malamud2016portfolio} develop a general equilibrium model demonstrating that ETF trading can affect the volatility and correlation structure of underlying assets through portfolio rebalancing effects. Their model shows that when ETFs experience large flows, the resulting trades in underlying assets can create systematic volatility spillovers that persist beyond the initial shock.

\citet{ben2017etfs} provide a comprehensive theoretical framework for understanding how ETFs affect price discovery in underlying markets. Their model demonstrates that ETFs can both enhance and impede price discovery, depending on the relative efficiency of ETF and underlying asset markets. When ETF markets are more liquid and efficient, they can serve as venues for informed trading that ultimately improves price discovery in underlying assets. Conversely, when ETF trading is dominated by uninformed flow, it can inject noise into underlying asset prices.

The theoretical literature has also examined the specific challenges facing commodity ETFs. \citet{basak2016model} develop a model of commodity financialization that demonstrates how financial trading through commodity ETFs can increase price volatility and reduce the correlation between prices and fundamental supply and demand factors. Their model shows that when financial traders dominate commodity markets, prices can deviate substantially from fundamental values, creating complex volatility transmission patterns.

\subsection{Empirical Evidence on ETF Market Efficiency and Pricing}

The empirical literature examining ETF market efficiency presents mixed evidence on the effectiveness of arbitrage mechanisms. \citet{petajisto2017inefficiencies} provides the most comprehensive analysis of ETF pricing efficiency using data on over 1,000 U.S. ETFs from 2000 to 2015. He documents significant and persistent pricing deviations, with median absolute deviations from NAV averaging 40 basis points for bond ETFs and 15 basis points for equity ETFs. Importantly, these deviations are not randomly distributed but exhibit systematic patterns related to fund characteristics, market conditions, and underlying asset liquidity.

\citet{shin2013basis} examine the determinants of ETF pricing errors using a comprehensive sample of international equity ETFs. They find that pricing errors are significantly related to underlying market volatility, ETF trading volume, and the liquidity of underlying assets. Their results suggest that arbitrage effectiveness deteriorates precisely when it is most needed—during periods of high volatility and market stress.

The role of authorized participants in maintaining ETF pricing efficiency has received considerable empirical attention. \citet{dannhauser2017effect} examines how the number and identity of authorized participants affects ETF pricing. Using detailed data on AP activity, she finds that ETFs with more authorized participants exhibit smaller and less persistent pricing errors. However, she also documents that during periods of market stress, even ETFs with multiple APs can experience significant pricing dislocations.

\citet{pan2016etf} provide compelling evidence on how liquidity mismatches between ETFs and underlying assets affect arbitrage effectiveness. Using the 2014-2015 Russian financial crisis as a natural experiment, they show that when underlying assets become illiquid, ETFs can trade at substantial premiums or discounts to NAV for extended periods. During the peak of the crisis, some Russian equity ETFs traded at discounts exceeding 20\% of NAV for several weeks.

The empirical literature has also examined ETF pricing during extreme market events. \citet{madhavan2012exchange} provides a detailed analysis of ETF behavior during the May 6, 2010 flash crash. He documents that many ETFs experienced severe price dislocations, with some trading at discounts exceeding 40\% of NAV. Importantly, these dislocations were not randomly distributed but were concentrated among ETFs holding less liquid underlying assets and those with fewer authorized participants.

\citet{staer2017asset} extends this analysis by examining ETF behavior during multiple crisis episodes, including the 2008 financial crisis, the 2010 European sovereign debt crisis, and the 2015 Chinese stock market crash. She finds that ETF pricing errors increase systematically during crisis periods, with the magnitude of errors related to underlying asset illiquidity and the complexity of the ETF structure.

Recent empirical work has focused on the high-frequency dynamics of ETF pricing. \citet{richie2008examination} examine intraday price relationships between the SPDR S\&P 500 ETF (SPY) and its underlying index using tick-by-tick data. They find evidence of rapid price discovery, with ETF prices adjusting to new information within minutes. However, they also document systematic patterns in pricing errors related to market opening and closing effects.

\citet{hasbrouck2003intraday} extends this analysis to examine price discovery across multiple ETFs and their underlying assets. Using vector error correction models applied to high-frequency data, he finds that price discovery is shared between ETF and underlying asset markets, with the relative contribution varying based on trading volume and market conditions. His results suggest that ETFs serve as important venues for price discovery, particularly during periods of high market activity.

\subsection{ETF Impact on Underlying Asset Markets: Price Discovery and Volatility}

The growing prominence of ETFs has fundamentally altered the microstructure of underlying asset markets, with important implications for price discovery and volatility transmission. The empirical literature has produced competing findings on whether ETFs enhance or impede market efficiency, reflecting the complex nature of these relationships.

\citet{glosten2021etf} provide strong evidence that ETFs enhance price discovery in underlying assets. Using comprehensive data on ETF trading and underlying stock prices, they show that ETF order flow contains significant information about future stock returns. They estimate that ETF trading accounts for approximately 15\% of total price discovery in large-cap stocks, with this contribution increasing during periods of high market volatility. Their findings suggest that ETFs serve as efficient aggregation mechanisms for broad market information.

However, \citet{ben2018etfs} present compelling counter-evidence suggesting that ETFs may increase non-fundamental volatility in underlying assets. Using a comprehensive sample of U.S. stocks from 1993 to 2012, they find that stocks with higher ETF ownership exhibit 15-20\% higher return volatility, even after controlling for fundamental volatility measures. This effect is particularly pronounced for stocks held by multiple ETFs, suggesting that ETF-driven correlations create systematic volatility spillovers across securities.

\citet{israeli2017etf} extend this analysis by examining how ETF ownership affects information production and processing in underlying markets. They find that stocks with higher ETF ownership receive less analyst coverage and exhibit weaker relationships between accounting fundamentals and stock prices. These findings suggest that while ETFs may improve market-level price discovery, they may simultaneously reduce security-specific information production.

The impact of ETFs on underlying asset liquidity represents another contested area in the literature. \citet{hamm2014liquidity} documents that increased ETF ownership is associated with reduced liquidity in individual stocks, as measured by wider bid-ask spreads and higher price impact measures. She argues that this occurs because traders substitute ETF trading for individual stock trading, reducing the flow of stock-specific information and decreasing market maker incentives to provide liquidity in individual securities.

Conversely, \citet{agarwal2018etfs} find that ETFs generally improve liquidity in underlying assets by attracting additional trading interest and providing more efficient price discovery mechanisms. Using a comprehensive sample of U.S. equity ETFs, they show that stocks with higher ETF ownership exhibit lower transaction costs and higher trading volume. They argue that ETFs create positive externalities by attracting uninformed traders who provide liquidity to informed traders.

The resolution of this apparent contradiction may lie in the heterogeneity of ETF effects across different market segments and time periods. \citet{krause2014exchange} examine ETF effects across different market capitalization segments, finding that ETFs improve liquidity for large-cap stocks but may reduce liquidity for small-cap stocks. They argue that this reflects the fixed costs of maintaining ETF arbitrage relationships, which make arbitrage less profitable for smaller, less liquid stocks.

Recent empirical work has examined the dynamic nature of ETF effects on underlying markets. \citet{da2018exchange} show that the impact of ETFs on underlying asset correlations varies significantly over time, with stronger effects during periods of market stress. They find that ETF-driven correlations increase substantially during crisis periods, potentially amplifying systemic risk and reducing diversification benefits.

\citet{broman2016liquidity} provide evidence on the style-investing effects of ETFs, showing that ETFs can create excess comovement among stocks with similar characteristics. Using data on sector and style ETFs, they demonstrate that stocks held by the same ETFs exhibit higher return correlations, even after controlling for fundamental similarities. This excess comovement represents a source of non-fundamental volatility that can persist for extended periods.

\subsection{High-Frequency Volatility Measurement and Modeling}

The development of realized variance measures using high-frequency data represents one of the most significant advances in financial econometrics over the past two decades. This literature provides the methodological foundation for our analysis of volatility transmission in ETF markets.

\citet{andersen2001distribution} provide the seminal contribution to this literature, demonstrating that realized variance computed from high-frequency returns provides nearly unbiased estimates of integrated volatility. Using comprehensive data on individual stocks and market indices, they show that 5-minute realized variance measures substantially outperform traditional volatility proxies based on daily data. Their work establishes the statistical properties of realized variance and demonstrates its superiority for volatility forecasting applications.

\citet{barndorff2002econometric} extend this framework by providing rigorous asymptotic theory for realized variance estimators. They demonstrate that as the sampling frequency increases, realized variance converges to integrated volatility under general conditions. Importantly, they also show that market microstructure noise can bias realized variance estimates when sampling frequencies become too high, establishing the trade-off between statistical efficiency and microstructure bias that guides empirical applications.

The presence of price jumps in high-frequency data creates additional complications for volatility measurement. \citet{barndorff2004power} address this challenge by developing bipower variation estimators that provide robust measures of integrated volatility in the presence of jumps. Their approach allows researchers to decompose total price variation into continuous and jump components, enabling more refined analysis of volatility dynamics.

\citet{huang2005using} extend jump detection methods by developing formal statistical tests for the presence of jumps in high-frequency data. Their tests are based on the difference between realized variance and bipower variation, with critical values determined by the asymptotic distribution theory. These methods have become standard tools for identifying and analyzing jump effects in financial markets.

The literature has also addressed the optimal choice of sampling frequency for realized variance estimation. \citet{hansen2005realized} provide comprehensive analysis of this question, showing that the optimal sampling frequency depends on the relative importance of statistical efficiency and microstructure bias. For most liquid assets, they recommend sampling frequencies between 5 and 20 minutes as providing the best balance between these competing considerations.

\citet{liu2015does} conduct an extensive comparison of realized variance estimators across multiple asset classes, examining over 400 different estimators applied to 31 financial assets. Surprisingly, they find little evidence that sophisticated estimators systematically outperform simple 5-minute realized variance in forecasting applications. This finding has important implications for practical applications, suggesting that the benefits of complex estimators may not justify their additional computational costs.

The Heterogeneous Autoregressive (HAR) model of \citet{corsi2009simple} has emerged as the leading framework for modeling realized volatility dynamics. The model's key insight is that volatility exhibits different degrees of persistence across multiple time horizons, reflecting the heterogeneous trading behavior of market participants operating over daily, weekly, and monthly frequencies. Despite its apparent simplicity, the HAR model consistently outperforms more complex alternatives, including GARCH and stochastic volatility models, in out-of-sample forecasting exercises.

\citet{andersen2007roughing} extend the HAR framework by incorporating jump components and examining their forecasting performance. They show that separately modeling continuous and jump components of volatility can improve forecasting accuracy, particularly at longer horizons. Their HAR-CJ model has become a standard specification for volatility forecasting applications.

Recent advances have incorporated spillover effects into the HAR framework. \citet{bubak2011volatility} develop HAR-X models that allow volatility in one market to affect volatility in another market. These models have proven particularly useful for analyzing volatility transmission in international equity markets and commodity futures, providing a flexible framework for examining cross-market dynamics.

The literature has also explored frequency-domain approaches to volatility analysis. \citet{barunik2018measuring} develop spectral methods for measuring volatility spillovers at different frequency bands, enabling researchers to distinguish between short-term and long-term transmission effects. Their approach provides complementary insights to time-domain methods and has proven particularly valuable for understanding the mechanisms underlying volatility transmission.

\subsection{Bayesian Methods for Volatility Modeling}

The application of Bayesian methods to volatility modeling has gained considerable momentum in recent years, driven by their ability to handle parameter uncertainty and incorporate prior information in high-dimensional settings. This literature provides important methodological tools for our analysis of volatility transmission in ETF markets.

\citet{koop2011forecasting} demonstrates the advantages of Bayesian VAR methods for volatility forecasting in high-dimensional settings. Using comprehensive data on macroeconomic and financial variables, he shows that BVAR models with appropriate prior specifications can effectively handle the parameter proliferation problems that plague classical VAR estimation. The Minnesota prior, which shrinks coefficients toward zero with greater shrinkage for higher-order lags and cross-equation terms, proves particularly effective in volatility applications.

\citet{carriero2015forecasting} extend Bayesian VAR methods to exchange rate forecasting, demonstrating their effectiveness in capturing time-varying relationships between variables. Their approach incorporates stochastic volatility and time-varying parameters, providing a flexible framework for modeling evolving market relationships. These extensions are particularly relevant for ETF applications, where market relationships may change significantly over time.

The development of spillover indices based on Bayesian VAR models represents another important methodological advance. \citet{diebold2012measuring} introduce the spillover index methodology for quantifying the magnitude and direction of volatility transmission across markets. Their approach, based on forecast error variance decompositions from VAR models, provides intuitive measures of spillover intensity that have been widely adopted in empirical applications.

\citet{barunik2018measuring} extend spillover analysis to the frequency domain, enabling researchers to examine spillover effects at different time horizons. Their spectral approach reveals that spillover patterns often differ substantially between short-term and long-term frequencies, providing important insights into the mechanisms underlying volatility transmission.

Recent work has focused on incorporating structural breaks and regime changes into Bayesian volatility models. \citet{clark2008forecasting} develop time-varying parameter BVAR models that allow coefficients to evolve gradually over time. Their approach captures structural changes in volatility relationships without requiring ex-ante specification of break dates, providing a flexible framework for modeling evolving market dynamics.

\citet{primiceri2005time} extends this framework by allowing the error covariance matrix to evolve over time, capturing changes in the volatility and correlation structure of residuals. This extension is particularly important for spillover analysis, as it allows for time-varying spillover relationships that may change with market conditions.

\subsection{Commodity Markets and Financialization}

Commodity markets exhibit several distinctive features that differentiate them from traditional financial markets and create unique challenges for volatility modeling. The growing financialization of commodity markets through ETFs and other financial instruments has added new layers of complexity that require careful analysis.

\citet{gorton2006facts} provide a comprehensive analysis of commodity futures markets, documenting their distinctive risk and return characteristics. They show that commodity futures exhibit low correlations with stocks and bonds, making them attractive for portfolio diversification. However, they also document substantial heterogeneity across commodity sectors, with energy and agricultural commodities exhibiting different volatility patterns than precious metals.

The introduction of commodity index funds and ETFs has fundamentally altered the structure of commodity markets. \citet{buyuksahin2014speculation} examine the impact of financial speculation on commodity markets, focusing on the role of commodity index funds. They find that increased financial participation has strengthened correlations between commodity prices and equity markets, particularly during periods of financial stress. This "financialization" effect has important implications for volatility transmission, as it suggests that commodity ETFs may serve as conduits for transmitting financial market volatility to physical commodity markets.

\citet{basak2016model} develop a theoretical model of commodity financialization that demonstrates how financial trading can increase price volatility and reduce the correlation between prices and fundamental supply and demand factors. Their model shows that when financial traders dominate commodity markets, prices can deviate substantially from fundamental values, creating complex volatility transmission patterns between financial and physical markets.

The empirical evidence on commodity financialization effects is mixed. \citet{buyuksahin2010fundamentals} examine whether financial speculation has increased commodity price volatility, finding limited evidence of systematic effects. However, they document significant heterogeneity across commodities, with some markets showing clear evidence of financialization effects while others remain largely unaffected.

\citet{singleton2014investor} provides a more nuanced analysis of financialization effects, examining how different types of financial flows affect commodity markets. He finds that passive index investment has relatively modest effects on commodity prices, while active speculation can create substantial price movements. These findings suggest that the specific nature of financial participation matters crucially for understanding volatility transmission effects.

The tracking performance of commodity ETFs has received considerable attention due to the complexities involved in replicating commodity exposure through futures contracts. \citet{todorov2021etf} analyzes the sources of tracking error in commodity ETFs, identifying rollover costs, contango effects, and management fees as primary contributors. He shows that these factors can generate systematic deviations between ETF returns and commodity price movements, creating opportunities for arbitrage and volatility transmission.

\citet{guo2015leveraged} examine the specific case of leveraged commodity ETFs, which aim to provide multiples of daily commodity returns. They document substantial volatility drag effects in these instruments, showing that daily rebalancing requirements create systematic differences between ETF performance and underlying commodity movements. These findings suggest that volatility transmission mechanisms may be particularly complex for commodity ETFs due to their reliance on derivatives and daily rebalancing requirements.

Recent research has begun to examine the high-frequency price relationships in commodity ETF markets. \citet{ozdurak2020price} analyze intraday price discovery between crude oil futures and the United States Oil Fund (USO), finding evidence of bidirectional price discovery with the futures market playing a dominant role. However, they document time-varying patterns in price leadership, suggesting that market conditions significantly affect volatility transmission mechanisms.

\subsection{ETF Market Microstructure During Crisis Periods}

The behavior of ETF markets during periods of financial stress provides important insights into the robustness of arbitrage mechanisms and the potential for volatility transmission effects. The literature examining ETF performance during crisis periods reveals systematic patterns that have important implications for market stability and regulation.

The COVID-19 pandemic provided a natural experiment for examining ETF behavior under extreme market conditions. \citet{ohara2021etf} analyze the microstructure of ETF arbitrage during the March 2020 market turmoil, finding that many ETFs experienced unprecedented pricing dislocations. Corporate bond ETFs were particularly affected, with some trading at discounts exceeding 10\% of NAV as underlying bond markets became increasingly illiquid.

\citet{dannhauser2020etf} extends this analysis by examining the role of authorized participants during the pandemic crisis. She finds that APs significantly reduced their arbitrage activity during the peak of the crisis, contributing to wider pricing spreads and increased volatility. Importantly, she shows that APs with stronger balance sheets were more likely to continue arbitrage activities, suggesting that financial constraints play a crucial role in determining arbitrage effectiveness during crisis periods.

The literature has also examined the spillover effects from ETF markets to underlying assets during crisis periods. \citet{staer2017asset} analyzes ETF behavior during multiple crisis episodes, finding that ETF pricing errors can create feedback effects that amplify volatility in underlying markets. When ETFs trade at significant discounts to NAV, the resulting creation and redemption activities can force sales of underlying assets at unfavorable prices, creating downward pressure on asset prices.

\citet{pan2017market} provide evidence on the systemic implications of ETF market disruptions. Using data from the 2015 Chinese stock market crash, they show that ETF arbitrage disruptions can contribute to broader market instability. When ETF creation and redemption mechanisms break down, the resulting forced trading in underlying assets can amplify market volatility and contribute to fire sale dynamics.

Recent research has examined the regulatory responses to ETF market disruptions during crisis periods. \citet{lynch2021market} analyze the Federal Reserve's intervention in ETF markets during the COVID-19 pandemic, showing that central bank purchases of ETF shares helped stabilize pricing and restore arbitrage effectiveness. Their findings suggest that ETF markets may require active policy intervention during extreme crisis periods.

\subsection{Gaps in Existing Literature and Positioning of Current Research}

Despite the substantial literature reviewed above, several critical gaps remain in our understanding of volatility transmission in commodity ETF markets. These gaps motivate our research questions and highlight the potential contributions of our analysis.

First, the existing literature lacks comprehensive analysis of high-frequency volatility transmission patterns in commodity ETF markets. While some studies examine daily or weekly relationships, the intraday dynamics that drive ETF pricing remain largely unexplored. This gap is particularly important given evidence that volatility transmission patterns can differ substantially across frequency domains \citep{barunik2018measuring}.

Second, most existing research focuses on equity ETFs, with limited attention to the unique characteristics of commodity markets. The few studies that examine commodity ETFs typically focus on single commodities or short time periods, preventing comprehensive analysis of cross-sectional differences in volatility transmission patterns. Given the distinctive features of different commodity sectors—including storage costs, seasonal patterns, and supply disruptions—systematic analysis across commodity types is essential.

Third, the literature has not adequately addressed the role of indicative NAV (iNAV) calculations in ETF pricing dynamics. Most studies rely on end-of-day NAV calculations that may not capture the complex intraday relationships that drive ETF pricing. The construction and analysis of high-frequency iNAV series represents an important methodological advance that enables more precise measurement of arbitrage relationships and volatility transmission mechanisms.

Fourth, existing research has not systematically examined whether different sampling frequencies reveal different aspects of volatility transmission mechanisms. While some studies suggest that sampling frequency affects volatility measurement \citep{hansen2005realized}, the specific implications for understanding ETF-underlying asset relationships remain unclear. This question has important practical implications for risk management and trading strategies.

Finally, the literature lacks systematic comparison of different econometric approaches for modeling volatility transmission in ETF markets. While HAR and BVAR models have been applied separately to various financial markets, their comparative performance in ETF applications remains unexplored. Understanding the relative strengths and limitations of different modeling approaches is crucial for developing robust empirical findings.

Our study addresses these gaps by providing comprehensive analysis of high-frequency volatility transmission across multiple commodity ETF markets using both HAR and BVAR methodologies. By constructing detailed iNAV series at multiple sampling frequencies and examining systematic differences across commodity types, we provide new insights into the mechanisms through which volatility transmits between commodity ETFs and their underlying assets. This analysis contributes to both the theoretical understanding of ETF pricing mechanisms and the practical challenges facing investors, market makers, and regulators in these rapidly evolving markets.

The testable hypotheses that emerge from this literature review center on several key questions. First, we test whether volatility transmission between commodity ETFs and their iNAVs exhibits bidirectional relationships ($H_1$) versus unidirectional patterns ($H_0$). Second, we examine whether high-frequency sampling reveals transmission mechanisms ($H_1$) that are not captured in daily analysis ($H_0$). Third, we test whether precious metals and energy commodities exhibit systematically different transmission patterns ($H_1$) versus homogeneous relationships ($H_0$). These hypotheses provide clear frameworks for empirical testing that contribute to knowledge regardless of the specific outcomes obtained.