%% Auteur: Simon-Pierre Boucher, Université Laval, Chapitre 2 : thèse de doctorat

\section{Introduction} 
The global exchange-traded fund (ETF) market has experienced remarkable growth over the past two decades, with assets under management exceeding \$10 trillion by 2023 \citep{petajisto2017inefficiencies}.  Within this landscape, commodity ETFs have emerged as increasingly important financial instruments, providing investors with accessible exposure to commodity markets without the complexities of direct futures trading or physical storage \citep{gorton2006facts}. These instruments have fundamentally altered the structure of commodity markets, introducing new channels for price discovery and volatility transmission that warrant careful empirical investigation.

The theoretical foundation for understanding ETF pricing rests on the arbitrage mechanism that links ETF prices to their underlying Net Asset Values (NAVs). Under ideal market conditions, authorized participants (APs) ensure that ETF prices remain closely aligned with their NAVs through creation and redemption processes \citep{ackert2000arbitrage}. However, this mechanism may become less effective during periods of market stress, when liquidity constraints and trading costs can lead to persistent deviations from fundamental values \citep{madhavan2012exchange}. The resulting price dynamics create complex volatility transmission patterns between ETFs and their underlying assets that have important implications for market efficiency and risk management.

Despite the growing importance of commodity ETFs, our understanding of high-frequency volatility dynamics between these instruments and their underlying assets remains limited. While extensive research exists on equity ETFs \citep{ben2018etfs, israeli2017etf}, and some studies examine commodity ETF tracking performance \citep{todorov2021etf}, few investigations focus specifically on the intraday volatility spillover mechanisms that characterize these markets. This gap is particularly significant given that commodity markets exhibit distinct characteristics—including seasonal patterns, storage constraints, and supply disruptions—that may generate unique volatility transmission dynamics.

This study addresses these limitations by conducting a comprehensive analysis of volatility dynamics in commodity ETF markets using high-frequency data and advanced econometric techniques. We construct indicative Net Asset Value (iNAV) series for four major commodity ETFs—representing gold (GLD), silver (SLV), crude oil (USO), and natural gas (UNG)—using minute-by-minute pricing data over a 13-year period from 2010 to 2023. Our methodological approach combines Heterogeneous Autoregressive (HAR) models, which capture the long-memory characteristics of realized volatility \citep{corsi2009simple}, with Bayesian Vector Autoregression (BVAR) techniques that accommodate time-varying interdependencies while addressing overfitting concerns \citep{koop2011forecasting}.

Our research questions are structured around testable hypotheses that contribute to the literature regardless of empirical outcomes. First, we examine whether volatility transmission between commodity ETFs and their iNAVs exhibits bidirectional relationships versus unidirectional patterns. Second, we investigate whether high-frequency sampling reveals volatility transmission mechanisms that are obscured in traditional daily analysis. Third, we test whether different commodity types—precious metals versus energy commodities—exhibit systematically different volatility transmission patterns. Finally, we assess whether the iNAV serves merely as a passive reflection of underlying asset movements or functions as an active determinant of ETF volatility.

Our empirical findings reveal significant heterogeneity in volatility transmission patterns across commodity types and sampling frequencies. For precious metals ETFs (GLD and SLV), we document predominantly unidirectional spillovers from iNAV to ETF, with coefficients ranging from 0.35 to 0.63 in our HAR specifications. Energy commodity ETFs (USO and UNG) exhibit bidirectional transmission, though with stronger effects flowing from iNAV to ETF markets. High-frequency (1-minute) sampling captures substantially stronger daily effects compared to lower frequencies, while weekly and monthly components remain stable across sampling intervals. These patterns are robust across both HAR and BVAR frameworks, with impulse response functions confirming rapid shock absorption and asymmetric adjustment mechanisms.

This study makes several important contributions to the literature. Methodologically, we are the first to construct comprehensive high-frequency iNAV series for commodity ETFs over an extended time period, enabling detailed analysis of intraday market dynamics. Empirically, we document systematic differences in volatility transmission mechanisms across commodity types, providing new insights into the role of underlying asset characteristics in determining ETF market behavior. From a practical perspective, our findings have important implications for risk management strategies, suggesting that monitoring underlying asset volatility is particularly crucial for precious metals ETFs, while energy ETF markets require attention to bidirectional feedback effects.

For investors, our results suggest that volatility forecasting models for commodity ETFs should incorporate high-frequency information from underlying assets, particularly for short-term trading strategies. For regulators, the documented asymmetries in volatility transmission indicate that oversight approaches may need to be tailored to different commodity types, with particular attention to the potential for amplified volatility effects in precious metals markets. For market makers and authorized participants, our findings highlight the importance of understanding frequency-dependent arbitrage opportunities and the varying effectiveness of price discovery mechanisms across different commodity sectors.

The remainder of this paper is structured as follows. Section 2 reviews the relevant literature on ETF pricing mechanisms, volatility transmission, and commodity market dynamics. Section 3 describes our data construction methodology and presents descriptive statistics. Section 4 outlines our econometric framework, including HAR and BVAR model specifications. Section 5 presents our empirical results, and Section 6 concludes with implications for theory and practice.
