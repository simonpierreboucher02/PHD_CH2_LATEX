%% Auteur: Simon-Pierre Boucher, Université Laval, Chapitre 2 : thèse de doctorat

\section{Conclusion} \label{sec}
This study provides comprehensive empirical analysis of volatility transmission mechanisms between commodity ETFs and their underlying assets using high-frequency data and advanced econometric techniques. Our findings contribute to three distinct yet interconnected areas of financial economics: the theoretical understanding of ETF pricing mechanisms, the empirical measurement of volatility spillovers in derivative markets, and the practical challenges facing investors and regulators in increasingly complex financial markets.
Our results fundamentally advance theoretical understanding of how arbitrage mechanisms operate in practice within ETF markets. The systematic asymmetries we document between precious metals and energy commodities provide new insights into the conditions under which arbitrage mechanisms effectively maintain market efficiency versus those where structural frictions limit arbitrage effectiveness. For precious metals ETFs, we find strong evidence of unidirectional volatility transmission from underlying assets to ETFs, with limited reverse causation. This pattern supports theoretical models where physical arbitrage constraints create directional frictions that limit the ability of ETF markets to influence underlying asset prices. Energy commodity ETFs demonstrate markedly different patterns characterized by bidirectional transmission with systematic asymmetries favoring underlying-to-ETF effects, consistent with theoretical models of futures-based ETF arbitrage where the derivative nature of underlying exposures creates more balanced arbitrage relationships while maintaining directional dominance reflecting fundamental supply and demand forces in physical commodity markets.
The strength of our frequency-dependent findings provides new theoretical insights into the temporal structure of arbitrage relationships. The dramatic strengthening of daily effects at higher sampling frequencies indicates that arbitrage mechanisms operate primarily through high-frequency channels, while weekly and monthly relationships remain stable across frequency domains. This temporal separation suggests that different economic forces govern short-term arbitrage activities versus longer-term volatility persistence, with important implications for theoretical models of market efficiency. Our jump component analysis reveals that discontinuous price movements create distinct transmission channels that operate differently from continuous volatility effects, indicating that theoretical models of ETF pricing should explicitly account for the differential transmission of continuous versus discontinuous volatility components.
This study introduces several methodological innovations that advance empirical research in ETF markets and high-frequency volatility analysis. Our construction of comprehensive indicative Net Asset Value (iNAV) series represents the first systematic attempt to build minute-by-minute NAV estimates for commodity ETFs over an extended time period, enabling precise measurement of real-time arbitrage relationships that were previously unobservable using end-of-day NAV calculations. The comparison of HAR and BVAR modeling approaches provides new insights into the relative advantages of different econometric frameworks for analyzing volatility transmission, while our systematic examination of sampling frequency effects reveals that the choice of temporal aggregation fundamentally alters empirical findings about volatility relationships. The decomposition of volatility into continuous and jump components demonstrates that different types of price movements create distinct transmission mechanisms, with jump effects universally exhibiting stronger and more persistent transmission than continuous effects.
Our findings have important practical implications for different categories of market participants. For investors, our results indicate that volatility forecasting models for commodity ETFs should incorporate high-frequency information from underlying assets, particularly for short-term trading strategies, though approaches should be tailored to specific market characteristics rather than applying uniform methodologies across all commodity sectors. The frequency-dependent nature of findings suggests that participants operating at different time horizons should focus on different sources of information, with high-frequency traders emphasizing real-time underlying asset information while longer-term investors can rely more heavily on established volatility patterns. For market makers and authorized participants, the extreme asymmetries in precious metals markets suggest that arbitrage opportunities primarily flow from underlying markets to ETFs, while energy markets offer more balanced bidirectional opportunities.
Our findings reveal systematic differences in market structure and arbitrage effectiveness that have important implications for regulatory oversight and policy development. The documented asymmetries in volatility transmission suggest that regulatory approaches should be tailored to different commodity types rather than applying uniform oversight standards across all ETF markets. The strength and persistence of jump transmission effects across all commodity types indicate that regulators should pay particular attention to how discrete price movements propagate across ETF and underlying markets, while our frequency-dependent findings suggest that regulatory monitoring systems should incorporate high-frequency surveillance capabilities to detect arbitrage breakdowns and market dislocations that may not be apparent in traditional daily reporting frameworks.
While our study provides comprehensive analysis of volatility transmission in commodity ETF markets, several limitations suggest important directions for future research. Our focus on four major commodity ETFs limits generalizability across the broader universe of commodity ETFs, and our analysis period may not capture all possible configurations of market stress and arbitrage breakdown. The methodological innovations introduced in this study, particularly the construction of high-frequency iNAV series, open promising research directions for application to equity ETFs, international funds, and fixed-income ETFs. Our jump component analysis suggests that future research should examine the specific sources and characteristics of jumps that create the strongest transmission effects.
This study demonstrates that volatility transmission between commodity ETFs and their underlying assets involves complex, asymmetric, and frequency-dependent mechanisms that vary systematically across commodity types and market structures. Our findings reject simple hypotheses of homogeneous or frequency-invariant relationships, revealing instead a nuanced picture of market dynamics that reflects the interaction between arbitrage mechanisms, market structure characteristics, and operational constraints. The theoretical implications extend beyond commodity ETF markets to broader questions about market efficiency and arbitrage effectiveness, while the practical implications are immediate and substantial for investors, risk managers, market makers, and regulators. The continued growth and evolution of ETF markets, combined with increasing availability of high-frequency data and computational resources, ensures that the research agenda initiated in this study will remain relevant for understanding these important and rapidly evolving markets.