%% Auteur: Simon-Pierre Boucher, Université Laval, Chapitre 2 : thèse de doctorat

\section{Data} \label{sec:data}

This section describes our comprehensive dataset construction methodology, sample selection criteria, and the econometric challenges addressed in building high-frequency indicative Net Asset Value (iNAV) series for commodity ETFs. Our approach represents a significant methodological advance over existing studies by constructing minute-by-minute iNAV estimates that enable precise measurement of arbitrage relationships and volatility transmission mechanisms across multiple sampling frequencies.

\subsection{Sample Selection and Data Sources}

Our analysis focuses on four major commodity ETFs that collectively represent over \$15 billion in assets under management and provide broad exposure to different commodity sectors. The sample selection is based on several criteria designed to ensure robust empirical analysis while maintaining representativeness across commodity types.

First, we select ETFs with substantial trading volume and market capitalization to ensure adequate liquidity for high-frequency analysis. All selected ETFs rank among the top five in their respective commodity categories by assets under management throughout our sample period. Second, we require continuous trading and complete data availability across our entire sample period from January 1, 2010, to January 1, 2023. This 13-year timeframe encompasses multiple market regimes, including the 2010-2012 European sovereign debt crisis, the 2014-2016 commodity price collapse, the 2020 COVID-19 pandemic, and the 2021-2022 inflation surge, providing comprehensive coverage of different volatility environments.

Third, we select ETFs with different underlying structures to examine how replication methodology affects volatility transmission. Our sample includes both physically-backed ETFs (GLD, SLV) and futures-based ETFs (USO, UNG), enabling analysis of how different tracking mechanisms influence pricing dynamics. Finally, we focus on single-commodity ETFs rather than broad commodity indices to isolate specific volatility transmission mechanisms without the confounding effects of cross-commodity correlations.

Our primary data source is Bloomberg Terminal, which provides comprehensive tick-by-tick price data for both ETFs and their underlying assets. Bloomberg's institutional-quality data includes trade prices, bid-ask quotes, and trading volumes with millisecond timestamps, enabling precise measurement of high-frequency price relationships. We supplement this with futures price data from the Chicago Mercantile Exchange (CME) and London Bullion Market Association (LBMA) for underlying asset pricing.

The **SPDR Gold Trust (GLD)** serves as our representative precious metals ETF, tracking the price of gold bullion held in trust. GLD is the world's largest gold ETF with over \$50 billion in assets under management and daily trading volume exceeding \$1 billion. The fund holds physical gold bars stored in secure vaults, making it a direct proxy for gold price movements. This physical backing structure minimizes tracking error but creates unique arbitrage dynamics due to the costs and complexities of physical gold delivery.

The **iShares Silver Trust (SLV)** provides exposure to silver price movements through physical silver holdings. SLV represents the largest silver ETF globally, with assets exceeding \$10 billion and average daily trading volume of approximately \$500 million. Like GLD, SLV's physical backing structure creates specific arbitrage mechanisms that differ from futures-based alternatives. Silver's dual role as both an industrial metal and store of value creates unique supply and demand dynamics that may generate distinctive volatility patterns.

The **United States Oil Fund (USO)** tracks West Texas Intermediate (WTI) crude oil prices through futures contracts rather than physical holdings. USO pioneered the commodity ETF structure and remains the largest oil-focused ETF with over \$3 billion in assets. The fund's futures-based structure requires continuous rolling of expiring contracts, creating tracking errors related to contango and backwardation effects \citep{todorov2021etf}. These mechanical trading requirements generate systematic volatility patterns that provide insights into derivative-based ETF dynamics.

The **United States Natural Gas Fund (UNG)** replicates natural gas price exposure through Henry Hub natural gas futures contracts. UNG represents the primary vehicle for natural gas investment, with assets exceeding \$1 billion despite significant tracking challenges. Natural gas markets exhibit extreme seasonality and storage constraints that create unique volatility patterns, making UNG an ideal case study for examining how fundamental market characteristics affect ETF pricing dynamics.

\subsection{High-Frequency Data Construction and Cleaning}

High-frequency financial data requires extensive cleaning and filtering to remove microstructure noise and ensure statistical reliability. Our data cleaning process follows established protocols in the literature while addressing specific challenges posed by ETF market structure \citep{barndorff2009realized}.

We begin with raw tick-by-tick data from Bloomberg, including all trades and quotes during regular U.S. market hours (9:30 AM to 4:00 PM EST). Our initial dataset contains over 50 million individual price observations across all ETFs and underlying assets. We apply several filters to ensure data quality and consistency.

First, we remove obvious outliers using the method of \citet{brownlees2014practical}, eliminating trades that deviate by more than 10 standard deviations from the rolling 20-minute median price. This filter removes approximately 0.03\% of observations, primarily consisting of obvious data errors or extreme outliers that would bias volatility estimates.

Second, we implement the duration-based filtering approach of \citet{hansen2005realized} to address irregular trading patterns. We exclude observations when the time between consecutive trades exceeds 30 minutes, as such gaps typically indicate market closures or technical issues. This filter affects less than 0.1\% of observations but ensures temporal consistency in our high-frequency series.

Third, we apply the \citet{lee1991inferring} algorithm to classify trades as buyer- or seller-initiated, enabling analysis of order flow effects on pricing. While not directly used in our volatility calculations, this classification provides valuable insights into the microstructure forces driving ETF pricing dynamics.

After cleaning, we construct synchronized price series at multiple sampling frequencies: 1-minute, 5-minute, and 30-minute intervals. Following \citet{andersen2001distribution}, we use previous-tick interpolation to align timestamps across different instruments, ensuring that our volatility calculations capture contemporaneous price relationships.

Our final cleaned dataset contains approximately 45 million price observations across all instruments and frequencies. The data loss from cleaning procedures is minimal and concentrated in periods of extremely low trading activity, ensuring that our results are not distorted by data quality issues.

\subsection{Indicative Net Asset Value (iNAV) Construction}

The construction of high-frequency indicative Net Asset Value (iNAV) series represents a major methodological contribution of our study. While end-of-day NAV calculations are straightforward, computing meaningful intraday NAV estimates requires sophisticated modeling of underlying asset pricing, currency effects, and operational complexities.

For physically-backed ETFs (GLD and SLV), iNAV construction follows the composition-based approach:

\begin{equation}
\text{iNAV}_{t} = \frac{1}{N_t} \left[ \text{Cash}_t + \sum_{i} \left( P_{it} \cdot f_{it} \cdot q_{it} \cdot c_{it} \right) \right]
\end{equation}

where $\text{iNAV}_t$ represents the indicative NAV at time $t$, $N_t$ is the number of outstanding ETF shares, $\text{Cash}_t$ represents the fund's cash holdings, $P_{it}$ is the price of underlying asset $i$ in local currency, $f_{it}$ is the currency conversion factor, $q_{it}$ is the quantity of asset $i$ held by the fund, and $c_{it}$ represents adjustment factors for accrued interest, dividends, or other cash flows.

For GLD, we use London Bullion Market Association (LBMA) gold prices converted to U.S. dollars using real-time foreign exchange rates. The fund's physical gold holdings are updated daily based on creation and redemption activity, while intraday changes reflect only price movements. For SLV, we employ similar methodology using LBMA silver prices and daily holdings updates.

For futures-based ETFs (USO and UNG), iNAV construction follows a more complex futures position model:

\begin{equation}
\text{iNAV}_{t} = \frac{1}{N_t} \left[ \text{Cash}_t + \sum_{j} \left( F_{jt} \cdot cc_{jt} \cdot q_{jt} \cdot m_{jt} \right) \right] \cdot FX_t
\end{equation}

where $F_{jt}$ represents the price of futures contract $j$ at time $t$, $cc_{jt}$ is the contract conversion factor, $q_{jt}$ is the number of contracts held, $m_{jt}$ is the contract multiplier, and $FX_t$ accounts for any currency conversion requirements.

For USO, we use real-time WTI crude oil futures prices from the New York Mercantile Exchange (NYMEX), typically focusing on the nearest-month contract while accounting for the fund's actual contract positions. The fund's portfolio composition changes due to monthly contract rolling, which we model using the fund's published rolling schedule and historical patterns.

For UNG, we employ Henry Hub natural gas futures prices from NYMEX, with particular attention to the fund's complex contract structure. Natural gas futures exhibit extreme seasonality and volatility, requiring careful modeling of the relationship between different contract months.

A critical challenge in iNAV construction involves accounting for the temporal mismatch between ETF trading hours and underlying asset market hours. While U.S. ETFs trade from 9:30 AM to 4:00 PM EST, underlying commodity markets operate on different schedules. Gold and silver trade nearly 24 hours through London and Asian markets, while oil and natural gas futures have specific trading sessions.

We address this challenge using the approach of \citet{krehbiel2019real}, employing price discovery weights to combine information from different market sessions. When underlying markets are closed, we use the most recent available prices adjusted for any relevant carry costs or storage considerations. For overnight periods, we incorporate information from international markets where available.

To validate our iNAV construction methodology, we compare our estimates to published end-of-day NAV values provided by fund sponsors. Our iNAV estimates exhibit correlation coefficients exceeding 0.999 with official NAV calculations, with mean absolute deviations below 5 basis points. These validation tests confirm the accuracy of our construction methodology and support the reliability of our high-frequency volatility analysis.

\subsection{Realized Variance Construction and Jump Detection}

Following the established literature on realized volatility measurement, we construct realized variance (RV) estimates using the sum of squared intraday returns \citep{andersen2001distribution}. For each asset $i$ and day $t$, realized variance is calculated as:

\begin{equation}
RV_{i,t} = \sum_{j=1}^{M} r_{i,t,j}^2
\end{equation}

where $r_{i,t,j}$ represents the $j$-th intraday return on day $t$ and $M$ is the number of intraday observations. We construct realized variance measures at three sampling frequencies: 1-minute (M=390), 5-minute (M=78), and 30-minute (M=13) intervals.

The choice of sampling frequency involves a fundamental trade-off between statistical efficiency and microstructure bias \citep{hansen2005realized}. Higher sampling frequencies provide more precise volatility estimates but are increasingly contaminated by bid-ask bounce and other microstructure effects. We examine multiple frequencies to assess the robustness of our volatility transmission findings across different temporal horizons.

To separate continuous price movements from discrete jumps, we employ the bipower variation methodology of \citet{barndorff2004power}. Realized bipower variation is calculated as:

\begin{equation}
BV_{i,t} = \mu_1^{-2} \sum_{j=2}^{M} |r_{i,t,j}| \cdot |r_{i,t,j-1}|
\end{equation}

where $\mu_1 = \sqrt{2/\pi} \approx 0.798$ is a scaling constant. Under the null hypothesis of no jumps, bipower variation provides a consistent estimator of integrated variance that is robust to the presence of jumps in the price process.

We identify significant jumps using the test statistic of \citet{huang2005using}:

\begin{equation}
J_{i,t} = \max(RV_{i,t} - BV_{i,t}, 0)
\end{equation}

The jump component $J_{i,t}$ captures the contribution of discontinuous price movements to total daily variance, while the continuous component $C_{i,t} = RV_{i,t} - J_{i,t}$ reflects smooth price evolution.

Jump detection plays a crucial role in our analysis because jumps may transmit differently across markets than continuous volatility. For example, jumps in underlying commodity prices due to supply disruptions or geopolitical events may generate immediate jumps in ETF prices through arbitrage mechanisms, while continuous volatility may exhibit more gradual transmission patterns.

\subsection{Descriptive Statistics and Stylized Facts}

Tables \ref{tab:desc_stats_5min}, \ref{tab:desc_stats_1min}, and \ref{tab:desc_stats_30min} present comprehensive descriptive statistics for our realized variance, bipower variation, and jump components across all three sampling frequencies. These statistics reveal several important stylized facts about commodity ETF volatility dynamics.

First, realized variance exhibits substantial heterogeneity across commodities and between ETFs and their underlying iNAVs. Crude oil (USO) displays the highest average volatility, with 5-minute realized variance averaging 0.081\% for iNAV and 0.066\% for the ETF. This asymmetry suggests that the underlying commodity exhibits higher volatility than the ETF, consistent with the dampening effects of arbitrage mechanisms and ETF market making.

Second, natural gas (UNG) shows the most extreme volatility patterns, with maximum daily realized variance exceeding 6\% for both ETF and iNAV. These extreme values reflect the structural characteristics of natural gas markets, including seasonal demand patterns, storage constraints, and weather-related supply disruptions. The similarity between ETF and iNAV volatility for natural gas suggests that arbitrage mechanisms may be less effective in smoothing volatility for this commodity.

Third, precious metals ETFs (GLD and SLV) exhibit more modest volatility levels but show interesting asymmetries between ETF and iNAV patterns. Gold shows higher ETF volatility than iNAV volatility on average, with some extreme outliers in ETF volatility (maximum exceeding 78\%). These patterns likely reflect the operational complexities of authorized participant arbitrage in physical gold markets, where settlement and delivery constraints may limit arbitrage effectiveness.

The jump components reveal systematic differences across commodities and market structures. Crude oil shows substantial jump activity in both ETF and iNAV series, with jumps accounting for approximately 15-20\% of total variance on average. This pattern is consistent with the sensitivity of oil markets to geopolitical events and supply disruptions that generate discrete price movements.

Conversely, precious metals show relatively smaller jump components, suggesting that gold and silver prices evolve more continuously. However, the presence of extreme outliers in ETF jump components suggests that ETF markets may occasionally experience technical disruptions or liquidity events that create artificial jump behavior.

The comparison across sampling frequencies reveals the expected pattern of decreasing average volatility but increasing maximum volatility as sampling frequency increases. This pattern reflects the trade-off between capturing more price variation (higher frequency) and avoiding microstructure noise (lower frequency). The stability of our key findings across different sampling frequencies provides confidence in the robustness of our volatility transmission results.

Figures \ref{fig:rv_uso}, \ref{fig:rv_gld}, \ref{fig:rv_slv}, and \ref{fig:rv_ung} present time series plots of realized variance for each commodity across all three sampling frequencies. These plots reveal several important temporal patterns in volatility dynamics.

The time series exhibit clear clustering effects, with periods of high volatility followed by periods of low volatility. This clustering is particularly pronounced during crisis periods, including the 2014-2016 oil price collapse (visible in USO), the 2016 Brexit referendum effects (visible in precious metals), and the 2020 COVID-19 pandemic (visible across all commodities).

The synchronization between ETF and iNAV volatility varies across commodities and time periods. Precious metals show generally high synchronization, with ETF and iNAV volatility moving closely together. Energy commodities show more complex patterns, with periods of close synchronization interrupted by episodes of divergence.

These temporal patterns motivate our econometric analysis of volatility transmission mechanisms. The time-varying nature of ETF-iNAV relationships suggests that simple correlation analysis would miss important dynamic features that require more sophisticated modeling approaches.

\subsection{Data Quality Assessment and Robustness Checks}

To ensure the reliability of our empirical results, we conduct extensive data quality assessments and robustness checks. These procedures address potential concerns about data errors, structural breaks, and sampling effects that could bias our volatility transmission estimates.

First, we examine the time series properties of our key variables using augmented Dickey-Fuller tests for unit roots. All realized variance series are strongly stationary in levels, eliminating concerns about spurious regression in our subsequent analysis. The log-transformed realized variance series exhibit slightly different persistence properties but remain stationary, supporting our use of HAR and BVAR modeling frameworks.

Second, we test for structural breaks in our volatility series using the methodology of \citet{bai2003computation}. We identify several significant break points corresponding to major market events, including the 2014 oil price collapse, the 2016 Brexit referendum, and the 2020 COVID-19 pandemic. However, our main volatility transmission relationships remain stable across these different regimes, suggesting that our findings reflect fundamental market mechanisms rather than period-specific effects.

Third, we assess the sensitivity of our results to alternative data construction choices. We re-estimate our key models using different sampling frequencies, alternative outlier filters, and different interpolation methods for missing observations. Our core findings remain robust across these alternative specifications, providing confidence in the reliability of our main results.

Finally, we compare our constructed iNAV series to alternative NAV proxies where available. For example, we compare our GLD iNAV estimates to the SPDR Gold Shares indicative optimized portfolio value (IOPV) published by the NYSE. Our estimates exhibit correlation coefficients exceeding 0.995 with official IOPV calculations, confirming the accuracy of our construction methodology.

These quality assessment procedures provide strong evidence for the reliability and robustness of our dataset. The combination of institutional-quality raw data, rigorous cleaning procedures, and extensive validation checks ensures that our empirical findings reflect genuine market phenomena rather than data artifacts or methodological biases.

\subsection{Comparative Analysis with Existing Studies}

Our dataset construction represents several advances over existing studies of ETF volatility dynamics. Most prior research relies on daily data that may miss important high-frequency transmission mechanisms \citep{ben2018etfs, israeli2017etf}. By constructing minute-by-minute price series, we capture volatility dynamics that operate at much shorter time horizons than previously analyzed.

Additionally, most existing studies use end-of-day NAV calculations that may not accurately reflect intraday arbitrage relationships \citep{petajisto2017inefficiencies}. Our construction of high-frequency iNAV series enables precise measurement of real-time pricing relationships and arbitrage opportunities.

The scope of our analysis also exceeds most prior studies. While previous research typically focuses on individual ETFs or short time periods, our comprehensive dataset spanning 13 years and multiple commodity types enables systematic analysis of cross-sectional differences in volatility transmission mechanisms.

The methodological rigor of our data construction procedures also represents an advance over existing literature. Our extensive cleaning, validation, and robustness checking procedures ensure that our findings reflect genuine economic phenomena rather than data quality issues or methodological artifacts.

These advances in data construction and methodology enable us to provide new insights into the fundamental mechanisms underlying volatility transmission in commodity ETF markets. The resulting empirical findings contribute both to academic understanding of ETF market dynamics and to the practical challenges facing investors, market makers, and regulators in these rapidly evolving markets.