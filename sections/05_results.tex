%% Auteur: Simon-Pierre Boucher, Université Laval, Chapitre 2 : thèse de doctorat

\section{Empirical Results} \label{sec:results}

This section presents our comprehensive empirical analysis of volatility transmission mechanisms between commodity ETFs and their underlying assets. We organize our findings around the testable hypotheses developed in Section 2, examining the direction and magnitude of volatility spillovers across different commodity types, sampling frequencies, and market conditions. Our results provide new insights into the fundamental mechanisms underlying ETF pricing dynamics and reveal systematic patterns that have important implications for market efficiency, risk management, and regulatory oversight.

\subsection{Sampling Frequency Effects and Temporal Dynamics in HAR Models}

Our analysis of HAR-X model specifications across multiple sampling frequencies reveals significant heterogeneity in volatility transmission patterns, providing strong evidence against the null hypothesis of frequency-invariant relationships. Tables \ref{tab:HAR_5min}, \ref{tab:HAR_1min}, and \ref{tab:HAR_30min} present the HAR-X model results for 5-minute, 1-minute, and 30-minute realized variance respectively, demonstrating that sampling frequency fundamentally alters our understanding of volatility dynamics.

At the 1-minute sampling frequency (Table \ref{tab:HAR_1min}), immediate volatility effects exhibit substantially stronger magnitudes compared to lower-frequency alternatives. For crude oil markets, the daily NAV volatility coefficient reaches 0.3715, representing a 29\% increase over the 5-minute estimate of 0.2878 (Table \ref{tab:HAR_5min}) and nearly tripling the 30-minute estimate of 0.1334 (Table \ref{tab:HAR_30min}). This pattern suggests that high-frequency sampling captures market reactions and information processing that occur within minutes of underlying price movements, consistent with the rapid arbitrage mechanisms that theoretically govern ETF pricing.

More remarkably, while daily effects strengthen dramatically with higher sampling frequency, weekly and monthly components exhibit remarkable stability across all frequency specifications. As shown across Tables \ref{tab:HAR_5min}--\ref{tab:HAR_30min}, the weekly NAV coefficient for crude oil remains consistently around 0.38--0.41 across all frequencies, while monthly components vary by less than 10\% across specifications. This stability indicates that longer-term volatility patterns reflect fundamental economic relationships that operate independently of measurement frequency, while short-term dynamics are highly sensitive to the temporal resolution of analysis.

The frequency sensitivity patterns vary systematically across commodity types in ways that reveal underlying market structure differences. Precious metals markets exhibit the most pronounced frequency effects, with gold showing daily NAV coefficients ranging from 0.6320 at 1-minute frequency (Table \ref{tab:HAR_1min}) to 0.3866 at 30-minute frequency (Table \ref{tab:HAR_30min}). This 64\% variation suggests that gold ETF arbitrage mechanisms operate primarily at very short time horizons, consistent with the high liquidity and continuous trading characteristics of global precious metals markets.

Energy commodity markets display more complex frequency dependencies that reflect their distinct operational characteristics. Natural gas shows the smallest frequency effects among all commodities studied, with daily coefficients varying from 0.1207 at 1-minute frequency (Table \ref{tab:HAR_1min}) to 0.0430 at 30-minute frequency (Table \ref{tab:HAR_30min}), representing only a 64\% variation compared to gold's much larger range. This relative stability likely reflects the structural features of natural gas markets, including storage constraints and pipeline capacity limitations that create longer-term adjustment processes in volatility transmission.

The cross-market transmission effects reveal equally striking patterns across frequencies. Table \ref{tab:HAR_5min} shows that at 5-minute frequency, crude oil ETF volatility has a modest impact on NAV volatility (0.1112), while at 1-minute frequency (Table \ref{tab:HAR_1min}), this effect decreases to 0.0895, and becomes even weaker at 30-minute frequency (0.1025 in Table \ref{tab:HAR_30min}). This frequency-dependent attenuation suggests that ETF-to-NAV transmission operates most effectively at intermediate time horizons.

\subsection{Commodity-Specific Asymmetries in Volatility Transmission}

Our examination of cross-sectional differences in volatility transmission patterns reveals systematic asymmetries that provide compelling evidence for rejecting the null hypothesis of homogeneous relationships across commodity types. Tables \ref{tab:HAR_5min}--\ref{tab:HAR_30min} demonstrate fundamental differences in market structure, arbitrage mechanisms, and information processing that create distinct volatility dynamics for precious metals versus energy commodities.

Precious metals markets exhibit predominantly unidirectional volatility transmission from iNAV to ETF, with limited evidence of reverse causation. For gold, as shown consistently across all tables, the daily iNAV effect on ETF volatility reaches 0.6320 at 1-minute frequency (Table \ref{tab:HAR_1min}), representing one of the strongest spillover coefficients observed in our entire analysis. In stark contrast, the impact of ETF volatility on iNAV dynamics remains economically and statistically insignificant, with coefficients of 0.0209 at 5-minute frequency (Table \ref{tab:HAR_5min}), -0.0008 at 1-minute frequency (Table \ref{tab:HAR_1min}), and -0.0146 at 30-minute frequency (Table \ref{tab:HAR_30min}).

This asymmetric pattern intensifies when examining lower sampling frequencies, where gold exhibits increasingly negative ETF-to-iNAV coefficients, reaching -0.1284 at 30-minute frequency (Table \ref{tab:HAR_30min}). The negative coefficients suggest that ETF volatility may actually serve as a volatility-dampening mechanism for underlying gold markets, consistent with theoretical models where ETF market makers provide liquidity that stabilizes underlying asset prices during periods of stress.

Silver markets demonstrate similar but somewhat attenuated asymmetric patterns. The daily iNAV coefficient on ETF volatility ranges from 0.4184 at 1-minute frequency (Table \ref{tab:HAR_1min}) to 0.2766 at 30-minute frequency (Table \ref{tab:HAR_30min}), while reverse effects remain consistently weak or negative across all specifications. At 5-minute frequency (Table \ref{tab:HAR_5min}), the ETF-to-NAV coefficient is -0.0086, becoming slightly positive (0.0041) at 1-minute frequency but turning negative again (-0.0340) at 30-minute frequency. The attenuation relative to gold likely reflects silver's dual role as both an industrial metal and store of value, creating additional sources of fundamental volatility that partially offset the dominance of financial market effects.

Energy commodity markets present markedly different patterns characterized by bidirectional volatility transmission with systematic asymmetries favoring iNAV-to-ETF effects. Crude oil demonstrates significant spillovers in both directions across all frequencies. At 1-minute frequency (Table \ref{tab:HAR_1min}), daily iNAV effects reach 0.3783 while ETF effects attain 0.0971. While both effects are statistically significant, the iNAV-to-ETF transmission is nearly four times larger in magnitude, indicating clear directional dominance despite bidirectional causation.

Natural gas exhibits the most balanced bidirectional transmission among all commodities studied, though iNAV effects still dominate. Comparing across tables, the daily iNAV coefficient reaches 0.1207 while the ETF coefficient attains 0.1238 at 1-minute frequency (Table \ref{tab:HAR_1min}), representing near-parity in transmission magnitudes. At 5-minute frequency (Table \ref{tab:HAR_5min}), the pattern shows 0.1051 for iNAV and 0.1264 for ETF effects, while at 30-minute frequency (Table \ref{tab:HAR_30min}), coefficients are 0.0811 and 0.0737 respectively. This relative balance across all frequencies likely reflects the extreme volatility and illiquidity characteristics of natural gas markets, where both ETF and underlying market participants face significant trading constraints that limit the effectiveness of arbitrage mechanisms.

The weekly and monthly transmission patterns visible in Tables \ref{tab:HAR_5min}--\ref{tab:HAR_30min} reinforce these cross-sectional differences while revealing additional insights into the persistence of volatility effects. Precious metals exhibit strong persistence in iNAV-to-ETF transmission at all horizons, with gold's weekly iNAV coefficient reaching 0.3292 across frequencies, suggesting that volatility transmission in precious metals markets exhibits sustained rather than decreasing persistence over time.

\subsection{Jump Components and Discontinuous Volatility Transmission}

The HAR-CJ-X model results presented in Tables \ref{tab:HAR_CJ_5min}, \ref{tab:HAR_CJ_1min}, and \ref{tab:HAR_CJ_30min} reveal complex relationships between continuous and discontinuous components of volatility that provide new insights into the mechanisms underlying volatility transmission. The decomposition of volatility into continuous (QPV) and jump (J) components demonstrates that transmission patterns differ fundamentally depending on the nature of underlying price movements, with important implications for understanding market efficiency and arbitrage effectiveness.

The continuous component (QPV) effects exhibit substantial heterogeneity across commodities and sampling frequencies, indicating that smooth price adjustments follow different transmission patterns than overall volatility. At 1-minute frequency (Table \ref{tab:HAR_CJ_1min}), gold displays significant negative coefficients for both iNAV and ETF continuous components, with magnitudes of -0.0331 and -0.1232 respectively. These negative relationships suggest that continuous price movements in precious metals markets may exhibit mean-reverting characteristics that dampen rather than amplify volatility transmission.

The pattern reverses dramatically at 30-minute frequency (Table \ref{tab:HAR_CJ_30min}), where gold's iNAV continuous component becomes significantly positive (0.0579), indicating that the relationship between continuous volatility components depends critically on the temporal horizon of analysis. This frequency-dependent sign reversal suggests that continuous volatility transmission mechanisms operate differently across short-term and medium-term time scales, potentially reflecting different types of trading activity and information processing.

Energy commodities demonstrate more stable continuous component relationships across frequencies, though with notable differences between crude oil and natural gas. Crude oil exhibits positive continuous effects at 5-minute frequency (Table \ref{tab:HAR_CJ_5min}) with QPV coefficients of 0.0491 for iNAV, while at 30-minute frequency (Table \ref{tab:HAR_CJ_30min}), the pattern shows positive but smaller effects. Natural gas presents the most stable continuous component patterns across frequencies, with consistently positive coefficients that strengthen at lower sampling frequencies, reaching 0.0572 at 30-minute frequency (Table \ref{tab:HAR_CJ_30min}).

The jump component results reveal dramatically different transmission patterns that highlight the importance of discontinuous price movements in ETF markets. Jump effects universally exhibit positive coefficients with substantially larger magnitudes than continuous effects, indicating that discrete price movements create stronger and more persistent volatility transmission than smooth price evolution.

Gold markets demonstrate the strongest jump transmission effects across all tables. At 1-minute frequency (Table \ref{tab:HAR_CJ_1min}), daily iNAV jumps generate coefficients of 0.4255, while at 5-minute frequency (Table \ref{tab:HAR_CJ_5min}), the coefficient reaches 0.2118. These effects persist strongly across weekly and monthly horizons, with weekly jump coefficients reaching 0.5091 at 1-minute frequency and monthly effects maintaining significance around 0.2341. The persistence of jump effects indicates that discrete price movements in gold markets create lasting volatility impacts that extend well beyond the initial shock period.

Silver exhibits similar but somewhat attenuated jump transmission patterns, with daily iNAV jump coefficients reaching 0.5098 at 1-minute frequency (Table \ref{tab:HAR_CJ_1min}) and 0.3747 at 5-minute frequency (Table \ref{tab:HAR_CJ_5min}). The strength of silver jump effects across all frequencies maintains the pattern observed in gold, though with some variation in magnitude that likely reflects silver's industrial demand component.

Energy commodity jump patterns reveal interesting differences between crude oil and natural gas that reflect their distinct market characteristics. Crude oil demonstrates strong jump transmission in both directions, with daily iNAV jump effects of 0.3061 and significant ETF jump effects of 0.0956 at 1-minute frequency (Table \ref{tab:HAR_CJ_1min}). At 5-minute frequency (Table \ref{tab:HAR_CJ_5min}), the corresponding effects are 0.2072 and 0.0944, indicating consistent bidirectional jump transmission across frequencies.

Natural gas exhibits weaker and less consistent jump transmission patterns compared to other commodities. At 1-minute frequency (Table \ref{tab:HAR_CJ_1min}), daily iNAV jump effects reach only 0.0924, while at 5-minute frequency (Table \ref{tab:HAR_CJ_5min}), the effects are negligible (-0.0027). This pattern likely reflects the extreme and idiosyncratic nature of natural gas price movements, where jumps may be driven by highly specific factors such as weather events or pipeline disruptions that do not translate directly into systematic volatility transmission relationships.

\subsection{Bayesian Vector Autoregression Analysis}

The BVAR results presented in Tables \ref{tab:VAR_USO}--\ref{tab:VAR_UNG} provide complementary evidence on volatility transmission mechanisms while revealing the temporal structure of dynamic relationships between ETF and iNAV volatility. The Bayesian framework enables precise estimation of coefficient distributions and credible intervals, providing robust statistical inference about the strength and persistence of volatility relationships.

The USO (crude oil) BVAR results in Table \ref{tab:VAR_USO} reveal strong autoregressive structures in both iNAV and ETF volatility dynamics, with first-order coefficients substantially larger than second-order effects. iNAV volatility exhibits a first-lag coefficient of 0.5704 with a narrow credible interval [0.5237, 0.6171], indicating strong persistence in underlying commodity volatility that operates primarily through short-term memory effects. The second-lag coefficient of 0.2606 [0.2206, 0.3005] suggests continued but diminishing persistence, consistent with the long-memory characteristics commonly observed in commodity volatility.

The cross-market effects in crude oil markets demonstrate clear asymmetric transmission patterns that reinforce our HAR model findings. ETF volatility exerts modest influence on iNAV dynamics, with a first-lag coefficient of 0.0735 [0.0343, 0.1135] that, while statistically significant, represents less than one-eighth the magnitude of iNAV self-persistence. Second-lag ETF effects become negligible (0.0156) with credible intervals [-0.0194, 0.0512] that include zero, indicating that ETF influence on underlying commodity volatility dissipates rapidly.

In contrast, iNAV volatility demonstrates substantial influence on ETF dynamics through both first and second lags. The first-lag coefficient of 0.2861 [0.2315, 0.3433] approaches the magnitude of ETF self-persistence (0.2941 [0.2460, 0.3418]), indicating that underlying commodity volatility rivals the ETF's own lagged volatility as a predictor of current ETF volatility. Second-lag effects remain economically significant at 0.1934 [0.1443, 0.2410], suggesting that iNAV volatility influences create persistent effects in ETF markets that extend beyond immediate responses.

The GLD (gold) markets results in Table \ref{tab:VAR_GLD} exhibit even stronger asymmetric patterns that provide compelling evidence for unidirectional transmission mechanisms. iNAV volatility demonstrates exceptional persistence with a first-lag coefficient of 0.5506 [0.4985, 0.6027] and significant second-lag effects of 0.2480 [0.2034, 0.2930]. These coefficients indicate that gold price volatility exhibits strong long-memory characteristics that create predictable patterns extending across multiple trading days.

The cross-market effects in gold markets demonstrate the most extreme asymmetry observed in our analysis. ETF volatility exhibits essentially no influence on iNAV dynamics, with a first-lag coefficient of -0.0101 [-0.0547, 0.0336] that is both economically negligible and statistically indistinguishable from zero. Second-lag ETF effects become slightly positive (0.0383 [0.0001, 0.0759]) but remain economically insignificant relative to iNAV self-persistence.

Conversely, iNAV volatility dominates ETF dynamics through both contemporaneous and lagged channels. The first-lag coefficient of 0.3487 [0.2848, 0.4126] exceeds ETF self-persistence (0.1549 [0.1004, 0.2083]) by more than double, indicating that underlying gold volatility is the primary driver of ETF volatility dynamics. Second-lag effects of 0.2758 [0.2221, 0.3307] remain substantial, suggesting that gold volatility influences persist strongly across multiple time periods.

SLV (silver) results in Table \ref{tab:VAR_SLV} demonstrate patterns intermediate between gold and crude oil, exhibiting strong asymmetric transmission but with somewhat more balanced relationships. iNAV volatility shows exceptional persistence with a first-lag coefficient of 0.6192 [0.5631, 0.6738], representing the strongest autoregressive effect observed across all commodities. Second-lag persistence of 0.2144 [0.1678, 0.2606] indicates continued memory effects, consistent with silver's role as both a precious metal and industrial commodity.

Cross-market effects in silver maintain the asymmetric pattern observed in other precious metals but with notable differences in magnitude and persistence. ETF influence on iNAV remains weak, with a slightly negative first-lag coefficient of -0.0446 [-0.0932, 0.0051] and modest positive second-lag effects of 0.0584 [0.0176, 0.0996]. While the second-lag effect achieves statistical significance, its economic magnitude remains small relative to iNAV self-persistence.

The reverse transmission from iNAV to ETF demonstrates strong effects that persist across multiple lags. The first-lag coefficient of 0.3760 [0.3102, 0.4404] substantially exceeds ETF self-persistence (0.1790 [0.1226, 0.2362]), while second-lag effects of 0.2016 [0.1471, 0.2563] maintain economic significance. These patterns indicate that silver volatility transmission operates through persistent channels that create lasting effects in ETF markets.

UNG (natural gas) results in Table \ref{tab:VAR_UNG} present the most complex volatility transmission patterns, exhibiting significant bidirectional effects with time-varying asymmetries that reflect the unique structural characteristics of natural gas markets. iNAV volatility demonstrates moderate persistence with a first-lag coefficient of 0.3919 [0.3488, 0.4366] and substantial second-lag effects of 0.2313 [0.1915, 0.2715]. The persistence structure indicates predictable volatility patterns while suggesting that natural gas markets may be subject to more frequent structural breaks than other commodities.

The bidirectional transmission patterns in natural gas markets provide the clearest evidence of mutual influence between ETF and underlying markets. ETF volatility exerts meaningful influence on iNAV dynamics through both first-lag effects of 0.0799 [0.0381, 0.1214] and stronger second-lag effects of 0.1567 [0.1189, 0.1938]. The increasing magnitude across lags suggests that ETF influences on underlying natural gas markets operate through delayed channels, possibly reflecting the operational complexities of natural gas trading and storage.

\subsection{Visual Analysis of Volatility Patterns and Dynamic Responses}

The realized volatility time series presented in Figures \ref{fig:rv_uso}--\ref{fig:rv_ung} provide visual confirmation of the systematic differences in volatility patterns across commodity types. Figure \ref{fig:rv_uso} shows that crude oil (USO) exhibits highly correlated volatility patterns between ETF and NAV, with both series displaying similar clustering patterns and synchronized spikes during periods of market stress. This visual evidence supports our statistical findings of bidirectional transmission in energy markets, where periods of elevated volatility tend to affect both markets simultaneously.

Figure \ref{fig:rv_gld} reveals that gold (GLD) markets display remarkably synchronized volatility patterns, with NAV volatility consistently leading ETF volatility during major spike episodes. The visual pattern strongly supports our econometric evidence of unidirectional transmission from underlying markets to ETF markets. The figure shows clear instances where NAV volatility increases precede corresponding increases in ETF volatility, with ETF movements appearing as delayed responses to underlying market developments.

Silver markets in Figure \ref{fig:rv_slv} show similar patterns to gold but with somewhat more complex dynamics, consistent with our findings of intermediate behavior between precious metals and energy commodities. The figure reveals periods where ETF and NAV volatility exhibit strong co-movement interspersed with episodes of temporary divergence, supporting our econometric evidence of predominantly unidirectional transmission with occasional feedback effects.

Natural gas markets in Figure \ref{fig:rv_ung} exhibit the most complex volatility patterns, with frequent divergences between ETF and NAV volatility that reflect the structural complexities of natural gas trading. The visual evidence supports our econometric findings of bidirectional but complex transmission mechanisms in natural gas markets, where episodes of joint volatility spikes alternate with periods of independent movement.

The impulse response function analysis presented in Figures \ref{fig:irf1}--\ref{fig:irf4} reveals dynamic insights into the temporal evolution of volatility shocks and their cross-market transmission patterns. Figure \ref{fig:irf1} shows that USO markets demonstrate rapid shock absorption with systematic asymmetries in cross-market transmission. The IRF plots reveal that iNAV shocks generate substantial and persistent responses in ETF volatility, while ETF shocks produce smaller and more transient effects on iNAV volatility, supporting our coefficient-based evidence of asymmetric bidirectional transmission.

Figure \ref{fig:irf2} reveals that GLD markets exhibit the most pronounced asymmetries in impulse response patterns, providing strong dynamic evidence for unidirectional transmission mechanisms. The figure shows that iNAV shocks create large and persistent responses in ETF volatility that decay smoothly over time, while ETF shocks generate negligible responses in iNAV volatility with complex oscillatory patterns that suggest limited and unstable transmission.

Silver markets in Figure \ref{fig:irf3} demonstrate impulse response patterns that maintain the asymmetric structure of gold markets while exhibiting more complex adjustment dynamics. The figure shows that while iNAV-to-ETF transmission remains dominant, the response patterns exhibit more variability than in gold markets, consistent with silver's dual role as precious metal and industrial commodity.

Natural gas markets in Figure \ref{fig:irf4} present the most complex impulse response patterns, reflecting the structural complexities of natural gas markets. The figure reveals substantial responses in both directions with complex temporal patterns that include delayed peaks and oscillatory behavior, supporting our findings of bidirectional transmission operating through multiple channels and time horizons.

\subsection{Synthesis and Hypothesis Testing}

Our comprehensive empirical analysis provides strong evidence for rejecting multiple null hypotheses while revealing systematic patterns in volatility transmission that advance theoretical understanding of ETF market dynamics. The evidence from Tables \ref{tab:HAR_5min}--\ref{tab:HAR_CJ_30min} and Tables \ref{tab:VAR_USO}--\ref{tab:VAR_UNG}, combined with the visual patterns in Figures \ref{fig:rv_uso}--\ref{fig:rv_ung} and \ref{fig:irf1}--\ref{fig:irf4}, supports a nuanced view of volatility transmission mechanisms that varies systematically across commodity types, sampling frequencies, and temporal horizons.

The evidence strongly rejects our first null hypothesis of homogeneous volatility transmission relationships across commodity types. Precious metals exhibit predominantly unidirectional transmission from iNAV to ETF, while energy commodities demonstrate bidirectional transmission with systematic asymmetries. These cross-sectional differences reflect fundamental variations in market structure, arbitrage mechanisms, and operational characteristics that create distinct volatility dynamics for different commodity sectors.

Our second null hypothesis of frequency-invariant relationships receives decisive rejection through evidence that sampling frequency fundamentally alters the measurement and understanding of volatility transmission mechanisms. The comparison across Tables \ref{tab:HAR_5min}--\ref{tab:HAR_30min} demonstrates that high-frequency data reveals transmission effects that are completely obscured in lower-frequency analysis, while different frequency domains capture different aspects of the underlying economic relationships.

The third null hypothesis of symmetric transmission patterns requires nuanced interpretation that depends critically on commodity type. For precious metals, the evidence from both HAR and BVAR models strongly supports asymmetric transmission patterns favoring iNAV-to-ETF effects, while energy commodities exhibit significant bidirectional relationships with systematic asymmetries still favoring iNAV-to-ETF transmission.

The decomposition of volatility into continuous and jump components in Tables \ref{tab:HAR_CJ_5min}--\ref{tab:HAR_CJ_30min} reveals that transmission mechanisms operate differently for smooth versus discontinuous price movements. Jump components consistently exhibit stronger and more persistent transmission effects than continuous components, indicating that discrete price movements create more significant volatility spillovers between ETF and underlying markets.

These empirical findings provide important insights into the fundamental mechanisms underlying ETF pricing and volatility dynamics. The systematic cross-sectional differences indicate that ETF market efficiency and arbitrage effectiveness depend critically on underlying market characteristics, with important implications for both theoretical models and practical applications in risk management and regulation. The frequency dependence of transmission effects highlights the importance of high-frequency analysis for understanding the operational mechanisms of ETF arbitrage, while the asymmetric patterns reveal fundamental differences in the efficiency of forward versus reverse arbitrage mechanisms across different commodity types.